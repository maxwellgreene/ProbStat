\documentclass[]{article}
\usepackage{lmodern}
\usepackage{amssymb,amsmath}
\usepackage{ifxetex,ifluatex}
\usepackage{fixltx2e} % provides \textsubscript
\ifnum 0\ifxetex 1\fi\ifluatex 1\fi=0 % if pdftex
  \usepackage[T1]{fontenc}
  \usepackage[utf8]{inputenc}
\else % if luatex or xelatex
  \ifxetex
    \usepackage{mathspec}
  \else
    \usepackage{fontspec}
  \fi
  \defaultfontfeatures{Ligatures=TeX,Scale=MatchLowercase}
\fi
% use upquote if available, for straight quotes in verbatim environments
\IfFileExists{upquote.sty}{\usepackage{upquote}}{}
% use microtype if available
\IfFileExists{microtype.sty}{%
\usepackage{microtype}
\UseMicrotypeSet[protrusion]{basicmath} % disable protrusion for tt fonts
}{}
\usepackage[margin=1in]{geometry}
\usepackage{hyperref}
\hypersetup{unicode=true,
            pdftitle={Max Greene HW10},
            pdfborder={0 0 0},
            breaklinks=true}
\urlstyle{same}  % don't use monospace font for urls
\usepackage{color}
\usepackage{fancyvrb}
\newcommand{\VerbBar}{|}
\newcommand{\VERB}{\Verb[commandchars=\\\{\}]}
\DefineVerbatimEnvironment{Highlighting}{Verbatim}{commandchars=\\\{\}}
% Add ',fontsize=\small' for more characters per line
\usepackage{framed}
\definecolor{shadecolor}{RGB}{248,248,248}
\newenvironment{Shaded}{\begin{snugshade}}{\end{snugshade}}
\newcommand{\KeywordTok}[1]{\textcolor[rgb]{0.13,0.29,0.53}{\textbf{#1}}}
\newcommand{\DataTypeTok}[1]{\textcolor[rgb]{0.13,0.29,0.53}{#1}}
\newcommand{\DecValTok}[1]{\textcolor[rgb]{0.00,0.00,0.81}{#1}}
\newcommand{\BaseNTok}[1]{\textcolor[rgb]{0.00,0.00,0.81}{#1}}
\newcommand{\FloatTok}[1]{\textcolor[rgb]{0.00,0.00,0.81}{#1}}
\newcommand{\ConstantTok}[1]{\textcolor[rgb]{0.00,0.00,0.00}{#1}}
\newcommand{\CharTok}[1]{\textcolor[rgb]{0.31,0.60,0.02}{#1}}
\newcommand{\SpecialCharTok}[1]{\textcolor[rgb]{0.00,0.00,0.00}{#1}}
\newcommand{\StringTok}[1]{\textcolor[rgb]{0.31,0.60,0.02}{#1}}
\newcommand{\VerbatimStringTok}[1]{\textcolor[rgb]{0.31,0.60,0.02}{#1}}
\newcommand{\SpecialStringTok}[1]{\textcolor[rgb]{0.31,0.60,0.02}{#1}}
\newcommand{\ImportTok}[1]{#1}
\newcommand{\CommentTok}[1]{\textcolor[rgb]{0.56,0.35,0.01}{\textit{#1}}}
\newcommand{\DocumentationTok}[1]{\textcolor[rgb]{0.56,0.35,0.01}{\textbf{\textit{#1}}}}
\newcommand{\AnnotationTok}[1]{\textcolor[rgb]{0.56,0.35,0.01}{\textbf{\textit{#1}}}}
\newcommand{\CommentVarTok}[1]{\textcolor[rgb]{0.56,0.35,0.01}{\textbf{\textit{#1}}}}
\newcommand{\OtherTok}[1]{\textcolor[rgb]{0.56,0.35,0.01}{#1}}
\newcommand{\FunctionTok}[1]{\textcolor[rgb]{0.00,0.00,0.00}{#1}}
\newcommand{\VariableTok}[1]{\textcolor[rgb]{0.00,0.00,0.00}{#1}}
\newcommand{\ControlFlowTok}[1]{\textcolor[rgb]{0.13,0.29,0.53}{\textbf{#1}}}
\newcommand{\OperatorTok}[1]{\textcolor[rgb]{0.81,0.36,0.00}{\textbf{#1}}}
\newcommand{\BuiltInTok}[1]{#1}
\newcommand{\ExtensionTok}[1]{#1}
\newcommand{\PreprocessorTok}[1]{\textcolor[rgb]{0.56,0.35,0.01}{\textit{#1}}}
\newcommand{\AttributeTok}[1]{\textcolor[rgb]{0.77,0.63,0.00}{#1}}
\newcommand{\RegionMarkerTok}[1]{#1}
\newcommand{\InformationTok}[1]{\textcolor[rgb]{0.56,0.35,0.01}{\textbf{\textit{#1}}}}
\newcommand{\WarningTok}[1]{\textcolor[rgb]{0.56,0.35,0.01}{\textbf{\textit{#1}}}}
\newcommand{\AlertTok}[1]{\textcolor[rgb]{0.94,0.16,0.16}{#1}}
\newcommand{\ErrorTok}[1]{\textcolor[rgb]{0.64,0.00,0.00}{\textbf{#1}}}
\newcommand{\NormalTok}[1]{#1}
\usepackage{graphicx,grffile}
\makeatletter
\def\maxwidth{\ifdim\Gin@nat@width>\linewidth\linewidth\else\Gin@nat@width\fi}
\def\maxheight{\ifdim\Gin@nat@height>\textheight\textheight\else\Gin@nat@height\fi}
\makeatother
% Scale images if necessary, so that they will not overflow the page
% margins by default, and it is still possible to overwrite the defaults
% using explicit options in \includegraphics[width, height, ...]{}
\setkeys{Gin}{width=\maxwidth,height=\maxheight,keepaspectratio}
\IfFileExists{parskip.sty}{%
\usepackage{parskip}
}{% else
\setlength{\parindent}{0pt}
\setlength{\parskip}{6pt plus 2pt minus 1pt}
}
\setlength{\emergencystretch}{3em}  % prevent overfull lines
\providecommand{\tightlist}{%
  \setlength{\itemsep}{0pt}\setlength{\parskip}{0pt}}
\setcounter{secnumdepth}{0}
% Redefines (sub)paragraphs to behave more like sections
\ifx\paragraph\undefined\else
\let\oldparagraph\paragraph
\renewcommand{\paragraph}[1]{\oldparagraph{#1}\mbox{}}
\fi
\ifx\subparagraph\undefined\else
\let\oldsubparagraph\subparagraph
\renewcommand{\subparagraph}[1]{\oldsubparagraph{#1}\mbox{}}
\fi

%%% Use protect on footnotes to avoid problems with footnotes in titles
\let\rmarkdownfootnote\footnote%
\def\footnote{\protect\rmarkdownfootnote}

%%% Change title format to be more compact
\usepackage{titling}

% Create subtitle command for use in maketitle
\newcommand{\subtitle}[1]{
  \posttitle{
    \begin{center}\large#1\end{center}
    }
}

\setlength{\droptitle}{-2em}

  \title{Max Greene HW10}
    \pretitle{\vspace{\droptitle}\centering\huge}
  \posttitle{\par}
    \author{}
    \preauthor{}\postauthor{}
    \date{}
    \predate{}\postdate{}
  

\begin{document}
\maketitle

\section{6.2.1}\label{section}

In problem 6.2.1 in the textbook, use the given information to compute
the \((1-\alpha)100\%\) confidence interval.

\subsection{a}\label{a}

\(H_0 : \mu=120\) versus
\(H_1 : \mu < 120 ; \bar{y}=114.2,n=25, \sigma =18, \alpha= 0.08\)

\begin{Shaded}
\begin{Highlighting}[]
\NormalTok{mu <-}\StringTok{ }\DecValTok{120} 
\NormalTok{y_bar <-}\StringTok{ }\FloatTok{114.2}
\NormalTok{n <-}\StringTok{ }\DecValTok{25}
\NormalTok{sigma <-}\StringTok{ }\DecValTok{18}
\NormalTok{alpha <-}\StringTok{ }\FloatTok{0.08}
\NormalTok{(z <-}\StringTok{ }\NormalTok{(y_bar}\OperatorTok{-}\NormalTok{mu)}\OperatorTok{/}\NormalTok{(sigma}\OperatorTok{/}\KeywordTok{sqrt}\NormalTok{(n)))}
\end{Highlighting}
\end{Shaded}

\begin{verbatim}
## [1] -1.611111
\end{verbatim}

\begin{Shaded}
\begin{Highlighting}[]
\NormalTok{(z_alpha_norm <-}\StringTok{ }\KeywordTok{qnorm}\NormalTok{(}\FloatTok{0.08}\NormalTok{))}
\end{Highlighting}
\end{Shaded}

\begin{verbatim}
## [1] -1.405072
\end{verbatim}

\begin{Shaded}
\begin{Highlighting}[]
\NormalTok{(z_alpha_test <-}\StringTok{ }\KeywordTok{qnorm}\NormalTok{(}\FloatTok{0.08}\NormalTok{))}\OperatorTok{*}\NormalTok{(sigma}\OperatorTok{/}\KeywordTok{sqrt}\NormalTok{(n))}\OperatorTok{+}\NormalTok{mu}
\end{Highlighting}
\end{Shaded}

\begin{verbatim}
## [1] 114.9417
\end{verbatim}

\subsection{b}\label{b}

\(H_0 : \mu=42.9\) versus
\(H_1 : \mu \ne 42.9 ; \bar{y}=45.1,n=16, \sigma =3.2, \alpha= 0.01\)

\begin{Shaded}
\begin{Highlighting}[]
\NormalTok{mu <-}\StringTok{ }\FloatTok{42.9}
\NormalTok{y_bar <-}\StringTok{ }\FloatTok{45.1}
\NormalTok{n <-}\StringTok{ }\DecValTok{16}
\NormalTok{sigma <-}\StringTok{ }\FloatTok{3.2}
\NormalTok{alpha <-}\StringTok{ }\FloatTok{0.01}
\NormalTok{(z <-}\StringTok{ }\NormalTok{(y_bar}\OperatorTok{-}\NormalTok{mu)}\OperatorTok{/}\NormalTok{(sigma}\OperatorTok{/}\KeywordTok{sqrt}\NormalTok{(n)))}
\end{Highlighting}
\end{Shaded}

\begin{verbatim}
## [1] 2.75
\end{verbatim}

\begin{Shaded}
\begin{Highlighting}[]
\NormalTok{(z_alpha_norm <-}\StringTok{ }\KeywordTok{qnorm}\NormalTok{(}\KeywordTok{c}\NormalTok{(alpha}\OperatorTok{/}\DecValTok{2}\NormalTok{,}\DecValTok{1}\OperatorTok{-}\NormalTok{alpha}\OperatorTok{/}\DecValTok{2}\NormalTok{)))}
\end{Highlighting}
\end{Shaded}

\begin{verbatim}
## [1] -2.575829  2.575829
\end{verbatim}

\begin{Shaded}
\begin{Highlighting}[]
\NormalTok{(z_alpha_test <-}\StringTok{ }\KeywordTok{qnorm}\NormalTok{(}\KeywordTok{c}\NormalTok{(alpha}\OperatorTok{/}\DecValTok{2}\NormalTok{,}\DecValTok{1}\OperatorTok{-}\NormalTok{alpha}\OperatorTok{/}\DecValTok{2}\NormalTok{)))}\OperatorTok{*}\NormalTok{(sigma}\OperatorTok{/}\KeywordTok{sqrt}\NormalTok{(n))}\OperatorTok{+}\NormalTok{mu}
\end{Highlighting}
\end{Shaded}

\begin{verbatim}
## [1] 40.83934 44.96066
\end{verbatim}

\subsection{c}\label{c}

\(H_0 : \mu=14.2\) versus
\(H_1 : \mu > 14.2 ; \bar{y}=15.8 , n=9, \sigma = 4.1 , \alpha= 0.13\)

\begin{Shaded}
\begin{Highlighting}[]
\NormalTok{mu <-}\StringTok{ }\FloatTok{14.2}
\NormalTok{y_bar <-}\StringTok{ }\FloatTok{15.8}
\NormalTok{n <-}\StringTok{ }\DecValTok{9}
\NormalTok{sigma <-}\StringTok{ }\FloatTok{4.1}
\NormalTok{alpha <-}\StringTok{ }\FloatTok{0.13}
\NormalTok{(z <-}\StringTok{ }\NormalTok{(y_bar}\OperatorTok{-}\NormalTok{mu)}\OperatorTok{/}\NormalTok{(sigma}\OperatorTok{/}\KeywordTok{sqrt}\NormalTok{(n)))}
\end{Highlighting}
\end{Shaded}

\begin{verbatim}
## [1] 1.170732
\end{verbatim}

\begin{Shaded}
\begin{Highlighting}[]
\NormalTok{(z_alpha_norm <-}\StringTok{ }\KeywordTok{qnorm}\NormalTok{(}\DecValTok{1}\OperatorTok{-}\NormalTok{alpha))}
\end{Highlighting}
\end{Shaded}

\begin{verbatim}
## [1] 1.126391
\end{verbatim}

\begin{Shaded}
\begin{Highlighting}[]
\NormalTok{(z_alpha_test <-}\StringTok{ }\KeywordTok{qnorm}\NormalTok{(}\DecValTok{1}\OperatorTok{-}\NormalTok{alpha))}\OperatorTok{*}\NormalTok{(sigma}\OperatorTok{/}\KeywordTok{sqrt}\NormalTok{(n))}\OperatorTok{+}\NormalTok{mu}
\end{Highlighting}
\end{Shaded}

\begin{verbatim}
## [1] 15.7394
\end{verbatim}

\section{6.2.2}\label{section-1}

Random sample, 22 diagnosed.

Past ADD score \(\mu_{ADD} = 95, \sigma_{ADD} = 15\).

Use \(\alpha = 0.06\). For what values of \(\bar{y}\) should \(H_0\) be
rejected if \(H_1\) is two-sided?

\begin{Shaded}
\begin{Highlighting}[]
\NormalTok{mu <-}\StringTok{ }\DecValTok{95}
\NormalTok{sigma <-}\StringTok{ }\DecValTok{15}
\NormalTok{n <-}\StringTok{ }\DecValTok{22}
\NormalTok{alpha <-}\StringTok{ }\FloatTok{0.06}
\NormalTok{(z_alpha_norm <-}\StringTok{ }\KeywordTok{qnorm}\NormalTok{(}\KeywordTok{c}\NormalTok{(alpha}\OperatorTok{/}\DecValTok{2}\NormalTok{,}\DecValTok{1}\OperatorTok{-}\NormalTok{alpha}\OperatorTok{/}\DecValTok{2}\NormalTok{)))}
\end{Highlighting}
\end{Shaded}

\begin{verbatim}
## [1] -1.880794  1.880794
\end{verbatim}

\begin{Shaded}
\begin{Highlighting}[]
\NormalTok{(z_alpha_test <-}\StringTok{ }\KeywordTok{qnorm}\NormalTok{(}\KeywordTok{c}\NormalTok{(alpha}\OperatorTok{/}\DecValTok{2}\NormalTok{,}\DecValTok{1}\OperatorTok{-}\NormalTok{alpha}\OperatorTok{/}\DecValTok{2}\NormalTok{)))}\OperatorTok{*}\NormalTok{(sigma}\OperatorTok{/}\KeywordTok{sqrt}\NormalTok{(n))}\OperatorTok{+}\NormalTok{mu}
\end{Highlighting}
\end{Shaded}

\begin{verbatim}
## [1]  88.9852 101.0148
\end{verbatim}

\section{6.2.4}\label{section-2}

Population mean 32,500. 15 sample drivers 33,800 on new tires. Assume
\(\sigma=4000\) for both cases. Can the company claim a statisticaly
significant difference at \(\alpha=0.05\)?

\begin{Shaded}
\begin{Highlighting}[]
\NormalTok{mu <-}\StringTok{ }\DecValTok{32500}
\NormalTok{sigma <-}\StringTok{ }\DecValTok{4000}
\NormalTok{n <-}\StringTok{ }\DecValTok{15}
\NormalTok{alpha <-}\StringTok{ }\FloatTok{0.05}
\NormalTok{(z_alpha_norm <-}\StringTok{ }\KeywordTok{qnorm}\NormalTok{(}\DecValTok{1}\OperatorTok{-}\NormalTok{alpha))}
\end{Highlighting}
\end{Shaded}

\begin{verbatim}
## [1] 1.644854
\end{verbatim}

\begin{Shaded}
\begin{Highlighting}[]
\NormalTok{mu}
\end{Highlighting}
\end{Shaded}

\begin{verbatim}
## [1] 32500
\end{verbatim}

\begin{Shaded}
\begin{Highlighting}[]
\NormalTok{(z_alpha_test <-}\StringTok{ }\KeywordTok{qnorm}\NormalTok{(}\DecValTok{1}\OperatorTok{-}\NormalTok{alpha))}\OperatorTok{*}\NormalTok{(sigma}\OperatorTok{/}\KeywordTok{sqrt}\NormalTok{(n))}\OperatorTok{+}\NormalTok{mu}
\end{Highlighting}
\end{Shaded}

\begin{verbatim}
## [1] 34198.8
\end{verbatim}

\section{6.2.10}\label{section-3}

Average blood pressure \(\mu = 120, \sigma = 12\). Sample of \(n=50\)
women, \(\mu = 125.2\).

\(H_0:\mu_{exam}\neq\mu_{norm}\) , \(H_1:\mu_{exam}=\mu_{norm}\)

\begin{Shaded}
\begin{Highlighting}[]
\NormalTok{mu <-}\StringTok{ }\DecValTok{120}
\NormalTok{sigma <-}\StringTok{ }\DecValTok{12}
\NormalTok{n <-}\StringTok{ }\DecValTok{50}
\NormalTok{alpha <-}\StringTok{ }\FloatTok{0.05}
\NormalTok{(z_alpha_norm <-}\StringTok{ }\KeywordTok{qnorm}\NormalTok{(}\DecValTok{1}\OperatorTok{-}\NormalTok{alpha))}
\end{Highlighting}
\end{Shaded}

\begin{verbatim}
## [1] 1.644854
\end{verbatim}

\begin{Shaded}
\begin{Highlighting}[]
\NormalTok{(z_alpha_test <-}\StringTok{ }\KeywordTok{qnorm}\NormalTok{(}\DecValTok{1}\OperatorTok{-}\NormalTok{alpha))}\OperatorTok{*}\NormalTok{(sigma}\OperatorTok{/}\KeywordTok{sqrt}\NormalTok{(n))}\OperatorTok{+}\NormalTok{mu}
\end{Highlighting}
\end{Shaded}

\begin{verbatim}
## [1] 122.7914
\end{verbatim}

\begin{Shaded}
\begin{Highlighting}[]
\NormalTok{alpha <-}\StringTok{ }\FloatTok{0.01}
\NormalTok{(z_alpha_norm <-}\StringTok{ }\KeywordTok{qnorm}\NormalTok{(}\DecValTok{1}\OperatorTok{-}\NormalTok{alpha))}
\end{Highlighting}
\end{Shaded}

\begin{verbatim}
## [1] 2.326348
\end{verbatim}

\begin{Shaded}
\begin{Highlighting}[]
\NormalTok{(z_alpha_test <-}\StringTok{ }\KeywordTok{qnorm}\NormalTok{(}\DecValTok{1}\OperatorTok{-}\NormalTok{alpha))}\OperatorTok{*}\NormalTok{(sigma}\OperatorTok{/}\KeywordTok{sqrt}\NormalTok{(n))}\OperatorTok{+}\NormalTok{mu}
\end{Highlighting}
\end{Shaded}

\begin{verbatim}
## [1] 123.9479
\end{verbatim}

This data shows a significant increase in heartrate with 99\%
confidence.

\section{6.3.2}\label{section-4}

\(67\)\% A/J are right-pawed.\\
Same protocol on \(n=35\) mice of strain A/HeJ, \(18/35\) right-pawed.
Test if A/HeJ mice were significantly different than A/J strains. Use
\(\alpha = 0.05\).

First check the inequality \[
0 < np_0-3\sqrt{np_0(1-p_0)}<np_0+3\sqrt{np_0(1-p_0)}<n
\]

\begin{Shaded}
\begin{Highlighting}[]
\NormalTok{n <-}\StringTok{ }\DecValTok{35}
\NormalTok{p_}\DecValTok{0}\NormalTok{ <-}\StringTok{ }\NormalTok{.}\DecValTok{67}
\DecValTok{0}
\end{Highlighting}
\end{Shaded}

\begin{verbatim}
## [1] 0
\end{verbatim}

\begin{Shaded}
\begin{Highlighting}[]
\NormalTok{n}\OperatorTok{*}\NormalTok{p_}\DecValTok{0}\OperatorTok{-}\DecValTok{3}\OperatorTok{*}\KeywordTok{sqrt}\NormalTok{(n}\OperatorTok{*}\NormalTok{p_}\DecValTok{0}\OperatorTok{*}\NormalTok{(}\DecValTok{1}\OperatorTok{-}\NormalTok{p_}\DecValTok{0}\NormalTok{))}
\end{Highlighting}
\end{Shaded}

\begin{verbatim}
## [1] 15.10455
\end{verbatim}

\begin{Shaded}
\begin{Highlighting}[]
\NormalTok{n}\OperatorTok{*}\NormalTok{p_}\DecValTok{0}\OperatorTok{+}\DecValTok{3}\OperatorTok{*}\KeywordTok{sqrt}\NormalTok{(n}\OperatorTok{*}\NormalTok{p_}\DecValTok{0}\OperatorTok{*}\NormalTok{(}\DecValTok{1}\OperatorTok{-}\NormalTok{p_}\DecValTok{0}\NormalTok{))}
\end{Highlighting}
\end{Shaded}

\begin{verbatim}
## [1] 31.79545
\end{verbatim}

\begin{Shaded}
\begin{Highlighting}[]
\NormalTok{n}
\end{Highlighting}
\end{Shaded}

\begin{verbatim}
## [1] 35
\end{verbatim}

\begin{Shaded}
\begin{Highlighting}[]
\NormalTok{n <-}\StringTok{ }\DecValTok{35}
\NormalTok{k <-}\StringTok{ }\DecValTok{18}
\NormalTok{p_}\DecValTok{0}\NormalTok{ <-}\StringTok{ }\NormalTok{.}\DecValTok{67}
\NormalTok{alpha <-}\StringTok{ }\FloatTok{0.05}
\NormalTok{(z <-}\StringTok{ }\NormalTok{(k}\OperatorTok{-}\NormalTok{n}\OperatorTok{*}\NormalTok{p_}\DecValTok{0}\NormalTok{)}\OperatorTok{/}\NormalTok{(}\KeywordTok{sqrt}\NormalTok{(n}\OperatorTok{*}\NormalTok{p_}\DecValTok{0}\OperatorTok{*}\NormalTok{(}\DecValTok{1}\OperatorTok{-}\NormalTok{p_}\DecValTok{0}\NormalTok{))))}
\end{Highlighting}
\end{Shaded}

\begin{verbatim}
## [1] -1.959152
\end{verbatim}

\begin{Shaded}
\begin{Highlighting}[]
\NormalTok{(z_alpha <-}\StringTok{ }\KeywordTok{qnorm}\NormalTok{(}\KeywordTok{c}\NormalTok{(alpha}\OperatorTok{/}\DecValTok{2}\NormalTok{,}\DecValTok{1}\OperatorTok{-}\NormalTok{alpha}\OperatorTok{/}\DecValTok{2}\NormalTok{)))}
\end{Highlighting}
\end{Shaded}

\begin{verbatim}
## [1] -1.959964  1.959964
\end{verbatim}

\section{6.3.4}\label{section-5}

Suppose \(H_0:p=0.45\) is to be tested against \(H_1:p>0.45\) at
\(\alpha=0.14\) level where \(p=P(i \text{th trial ends in success})\)
If sample size is \(n=200\) what is smallest number of successes that
will cause \(H_0\) to be rejected?

\begin{Shaded}
\begin{Highlighting}[]
\NormalTok{n <-}\StringTok{ }\DecValTok{200}
\NormalTok{k <-}\StringTok{ }\KeywordTok{c}\NormalTok{(}\DecValTok{79}\NormalTok{,}\DecValTok{80}\NormalTok{)}
\NormalTok{p_}\DecValTok{0}\NormalTok{ <-}\StringTok{ }\NormalTok{.}\DecValTok{45}
\NormalTok{alpha <-}\StringTok{ }\FloatTok{0.14}
\NormalTok{(z <-}\StringTok{ }\NormalTok{(k}\OperatorTok{-}\NormalTok{n}\OperatorTok{*}\NormalTok{p_}\DecValTok{0}\NormalTok{)}\OperatorTok{/}\NormalTok{(}\KeywordTok{sqrt}\NormalTok{(n}\OperatorTok{*}\NormalTok{p_}\DecValTok{0}\OperatorTok{*}\NormalTok{(}\DecValTok{1}\OperatorTok{-}\NormalTok{p_}\DecValTok{0}\NormalTok{))))}
\end{Highlighting}
\end{Shaded}

\begin{verbatim}
## [1] -1.563472 -1.421338
\end{verbatim}

\begin{Shaded}
\begin{Highlighting}[]
\NormalTok{(z_alpha <-}\StringTok{ }\KeywordTok{qnorm}\NormalTok{(}\KeywordTok{c}\NormalTok{(alpha}\OperatorTok{/}\DecValTok{2}\NormalTok{,}\DecValTok{1}\OperatorTok{-}\NormalTok{alpha}\OperatorTok{/}\DecValTok{2}\NormalTok{)))}
\end{Highlighting}
\end{Shaded}

\begin{verbatim}
## [1] -1.475791  1.475791
\end{verbatim}

\section{6.4.2}\label{section-6}

Carry out the details to verify the decision rule change cited on p.~364
in connection with Figure 6.4.6

???

\section{6.4.6}\label{section-7}

\(n=36, \sigma = 8\) for testing \(H_0 : \mu = 60\) versus
\(H_1 : \mu \neq 60\) at \(\alpha=0.07\).\\
Intends to reject \(H_0\) if \(\bar{y}\) falls in critical region region
\((60-\bar{y}^*,6-+\bar{y}^*)\)

\begin{enumerate}
\def\labelenumi{(\alph{enumi})}
\tightlist
\item
  Find \(\bar{y}^*\)
\end{enumerate}

\begin{Shaded}
\begin{Highlighting}[]
\NormalTok{mu <-}\StringTok{ }\DecValTok{60}
\NormalTok{n <-}\StringTok{ }\DecValTok{36}
\NormalTok{sigma <-}\StringTok{ }\DecValTok{8}
\NormalTok{alpha <-}\StringTok{ }\FloatTok{0.07}
\NormalTok{(z_alpha_norm <-}\StringTok{ }\KeywordTok{qnorm}\NormalTok{(}\KeywordTok{c}\NormalTok{(alpha}\OperatorTok{/}\DecValTok{2}\NormalTok{,}\DecValTok{1}\OperatorTok{-}\NormalTok{alpha}\OperatorTok{/}\DecValTok{2}\NormalTok{)))}
\end{Highlighting}
\end{Shaded}

\begin{verbatim}
## [1] -1.811911  1.811911
\end{verbatim}

\begin{Shaded}
\begin{Highlighting}[]
\NormalTok{(z_alpha_test <-}\StringTok{ }\KeywordTok{qnorm}\NormalTok{(}\KeywordTok{c}\NormalTok{(alpha}\OperatorTok{/}\DecValTok{2}\NormalTok{,}\DecValTok{1}\OperatorTok{-}\NormalTok{alpha}\OperatorTok{/}\DecValTok{2}\NormalTok{)))}\OperatorTok{*}\NormalTok{(sigma}\OperatorTok{/}\KeywordTok{sqrt}\NormalTok{(n))}
\end{Highlighting}
\end{Shaded}

\begin{verbatim}
## [1] -2.415881  2.415881
\end{verbatim}

\begin{enumerate}
\def\labelenumi{(\alph{enumi})}
\setcounter{enumi}{1}
\tightlist
\item
  Power of test when \(\mu=62\), true mean \(u_t=60\)?\\
  First calculate \(\beta\).
\end{enumerate}

\begin{Shaded}
\begin{Highlighting}[]
\NormalTok{mu <-}\StringTok{ }\DecValTok{62}
\NormalTok{muTrue <-}\StringTok{ }\DecValTok{60}
\NormalTok{n <-}\StringTok{ }\DecValTok{36}
\NormalTok{sigma <-}\StringTok{ }\DecValTok{8}
\NormalTok{alpha <-}\StringTok{ }\FloatTok{0.07}
\NormalTok{(Z_alpha <-}\StringTok{ }\NormalTok{(mu}\OperatorTok{-}\NormalTok{muTrue)}\OperatorTok{/}\NormalTok{(sigma}\OperatorTok{/}\KeywordTok{sqrt}\NormalTok{(n)))}
\end{Highlighting}
\end{Shaded}

\begin{verbatim}
## [1] 1.5
\end{verbatim}

\begin{Shaded}
\begin{Highlighting}[]
\NormalTok{(beta <-}\StringTok{ }\KeywordTok{pnorm}\NormalTok{(Z_alpha))}
\end{Highlighting}
\end{Shaded}

\begin{verbatim}
## [1] 0.9331928
\end{verbatim}

\begin{Shaded}
\begin{Highlighting}[]
\NormalTok{(power <-}\StringTok{ }\DecValTok{1}\OperatorTok{-}\NormalTok{beta)}
\end{Highlighting}
\end{Shaded}

\begin{verbatim}
## [1] 0.0668072
\end{verbatim}

\begin{enumerate}
\def\labelenumi{(\alph{enumi})}
\setcounter{enumi}{2}
\tightlist
\item
  Power of test when \(\mu=62\) if critical region defined correctly?
\end{enumerate}

???

\section{6.4.8}\label{section-8}

Will \(n=45\) be a sufficiently large sample to test \(H_0: \mu = 10\)
versus \(H_1 : \mu \neq 10\) at \(\alpha=0.05\) if experimenter wants
\(\beta < 0.20\) when \(\mu = 12\)? Assume \(\sigma=4\).

Using the guidelines from Ex. 6.4.1: \[
P(|Z| \geq 1.64)=0.05 \Rightarrow P(Z \geq 1.64)=0.025 \\
\bar{y}^*=10+1.64\frac{4}{\sqrt{n}}
\] Then, using power equation, \[
1-\beta = P(\text{reject }H_0 | H_1 \text{true})=P(\bar{Y}\geq \bar{y}^*|\mu=12)=0.80
\] Now, find associated \(Z_\alpha\)

\begin{Shaded}
\begin{Highlighting}[]
\NormalTok{beta <-}\StringTok{ }\FloatTok{0.20}
\KeywordTok{qnorm}\NormalTok{(beta)}
\end{Highlighting}
\end{Shaded}

\begin{verbatim}
## [1] -0.8416212
\end{verbatim}

So, \[
\begin{aligned}
\bar{y}^*&=12-0.84\frac{4}{\sqrt{n}}\\
10+1.64\frac{4}{\sqrt{n}}&=12-0.84\frac{4}{\sqrt{n}} \\
(1.64+0.84)\frac{4}{\sqrt{n}}&=2\\
n&=\Big(\frac{4*2.48}{2}\Big)^2 \approx 24.6
\end{aligned}
\]

Yes, a sample of \(n=40\) will be large enough to guarantee a power of
\(0.80\).

\section{6.4.12}\label{section-9}

Urn contains \(n=10\) chips with \(w=\) no. white and \(n-w=\) no.
red.\\
\(H_0:\) exactly half chips red, \(H_1:\) more than half red. Draw \(3\)
chips and reject \(H_0\) if two or more white. Find \(\alpha\). Then
find \(\beta\) when urn is \(60%\) white and \(70%\) white.

\[
\alpha=P(\bar{Y} > \bar{y}^* | H_0 \text{ true})
\]

Therefore, \(\alpha\) will be the probability that there are actually 5
white in total given either 2 or 3 out of the initial 3 being white. So,
this probability is the sum of the probabilities of 2 and 3 out of the
remaining 7 chips being white.

\begin{Shaded}
\begin{Highlighting}[]
\NormalTok{(alpha <-}\StringTok{ }\KeywordTok{sum}\NormalTok{(}\KeywordTok{dbinom}\NormalTok{(}\KeywordTok{c}\NormalTok{(}\DecValTok{2}\NormalTok{,}\DecValTok{3}\NormalTok{),}\DecValTok{7}\NormalTok{,}\DataTypeTok{prob=}\NormalTok{.}\DecValTok{5}\NormalTok{)))}
\end{Highlighting}
\end{Shaded}

\begin{verbatim}
## [1] 0.4375
\end{verbatim}

\begin{Shaded}
\begin{Highlighting}[]
\NormalTok{(beta60 <-}\StringTok{ }\DecValTok{1}\OperatorTok{-}\KeywordTok{sum}\NormalTok{(}\KeywordTok{dbinom}\NormalTok{(}\KeywordTok{c}\NormalTok{(}\DecValTok{2}\NormalTok{,}\DecValTok{3}\NormalTok{),}\DecValTok{7}\NormalTok{,}\DataTypeTok{prob=}\NormalTok{.}\DecValTok{6}\NormalTok{)))}
\end{Highlighting}
\end{Shaded}

\begin{verbatim}
## [1] 0.7290496
\end{verbatim}

\begin{Shaded}
\begin{Highlighting}[]
\NormalTok{(beta70 <-}\StringTok{ }\DecValTok{1}\OperatorTok{-}\KeywordTok{sum}\NormalTok{(}\KeywordTok{dbinom}\NormalTok{(}\KeywordTok{c}\NormalTok{(}\DecValTok{2}\NormalTok{,}\DecValTok{3}\NormalTok{),}\DecValTok{7}\NormalTok{,}\DataTypeTok{prob=}\NormalTok{.}\DecValTok{7}\NormalTok{)))}
\end{Highlighting}
\end{Shaded}

\begin{verbatim}
## [1] 0.8777548
\end{verbatim}

\section{6.4.14}\label{section-10}

Sample size \(n=1\) taken from pdf
\(f_Y(y)=(\theta+1)y^\theta, 0 \leq y \leq 1\). Reject \(H_0: \theta=1\)
for \(H_1:\theta > 1\) if \(y \geq 0.90\). What is test's level of
significance?


\end{document}
